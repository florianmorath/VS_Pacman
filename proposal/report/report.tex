% This is based on "sig-alternate.tex" V1.9 April 2009
% This file should be compiled with V2.4 of "sig-alternate.cls" April 2009
%
\documentclass{report}

\usepackage[english]{babel}
\usepackage{graphicx}
\usepackage{tabularx}
\usepackage{subfigure}
\usepackage{enumitem}
\usepackage{url}


\usepackage{color}
\definecolor{orange}{rgb}{1,0.5,0}
\definecolor{lightgray}{rgb}{.9,.9,.9}
\definecolor{java_keyword}{rgb}{0.37, 0.08, 0.25}
\definecolor{java_string}{rgb}{0.06, 0.10, 0.98}
\definecolor{java_comment}{rgb}{0.12, 0.38, 0.18}
\definecolor{java_doc}{rgb}{0.25,0.35,0.75}

% code listings

\usepackage{listings}
\lstloadlanguages{Java}
\lstset{
	language=Java,
	basicstyle=\scriptsize\ttfamily,
	backgroundcolor=\color{lightgray},
	keywordstyle=\color{java_keyword}\bfseries,
	stringstyle=\color{java_string},
	commentstyle=\color{java_comment},
	morecomment=[s][\color{java_doc}]{/**}{*/},
	tabsize=2,
	showtabs=false,
	extendedchars=true,
	showstringspaces=false,
	showspaces=false,
	breaklines=true,
	numbers=left,
	numberstyle=\tiny,
	numbersep=6pt,
	xleftmargin=3pt,
	xrightmargin=3pt,
	framexleftmargin=3pt,
	framexrightmargin=3pt,
	captionpos=b
}

% Disable single lines at the start of a paragraph (Schusterjungen)

\clubpenalty = 10000

% Disable single lines at the end of a paragraph (Hurenkinder)

\widowpenalty = 10000
\displaywidowpenalty = 10000
 
% allows for colored, easy-to-find todos

\newcommand{\todo}[1]{\textsf{\textbf{\textcolor{orange}{[[#1]]}}}}

% consistent references: use these instead of \label and \ref

\newcommand{\lsec}[1]{\label{sec:#1}}
\newcommand{\lssec}[1]{\label{ssec:#1}}
\newcommand{\lfig}[1]{\label{fig:#1}}
\newcommand{\ltab}[1]{\label{tab:#1}}
\newcommand{\rsec}[1]{Section~\ref{sec:#1}}
\newcommand{\rssec}[1]{Section~\ref{ssec:#1}}
\newcommand{\rfig}[1]{Figure~\ref{fig:#1}}
\newcommand{\rtab}[1]{Table~\ref{tab:#1}}
\newcommand{\rlst}[1]{Listing~\ref{#1}}

% General information

\title{Multiplayer Pacman\\
\normalsize{Distributed Systems -- Project Proposal}}
\subtitle{subtitle}

% Use the \alignauthor commands to handle the names
% and affiliations for an 'aesthetic maximum' of six authors.

\numberofauthors{1} %  in this sample file, there are a *total*
% of EIGHT authors. SIX appear on the 'first-page' (for formatting
% reasons) and the remaining two appear in the \additionalauthors section.
%
\author{
% You can go ahead and credit any number of authors here,
% e.g. one 'row of three' or two rows (consisting of one row of three
% and a second row of one, two or three).
%
% The command \alignauthor (no curly braces needed) should
% precede each author name, affiliation/snail-mail address and
% e-mail address. Additionally, tag each line of
% affiliation/address with \affaddr, and tag the
% e-mail address with \email.
%
% 1st. author
\alignauthor \normalsize{Student One,  Student Two, Student Three}\\
	\affaddr{\normalsize{ETH ID-1 XX-XXX-XXX, ETH ID-2 XX-XXX-XXX, ETH ID-3 XX-XXX-XXX}}\\
	\email{\normalsize{one@student.ethz.ch, two@student.ethz.ch, three@student.ethz.ch}}
}


\begin{document}

\maketitle

\section{Introduction}

%------------------------------------INTRODUCTION------------------------------------

Since 1980, the game Pac-Man has fascinated players around the world. 
Starting as an arcade game, it was adapted for many platforms while technology was improving and is still played today on mobile devices.
While graphics have improved significantly over time and new features were added to the original idea of the game, there is still a major drawback to most of its versions as the game only provides single player user experience.

 
Our goal is therefore to exploit the opportunities of modern mobile devices in order to create a new user experience. 
This new user experience will consist of the well known PacMan game combined with a distributed multi player approach.
The idea will be implemented as an android application allowing the user to create a new instance of the game by starting a server on an android device. 
Other users can then connect to this server with their own devices.



\section{System Overview}

%-----------------------------------SYSTEM OVERVIEW----------------------------------


\begin{figure}[h]
	\centering
    \includegraphics[width=\columnwidth]{example}
    \lfig{example}
    \vspace{-5mm} % use negative white space to fix too large gaps
	\caption{Only include useful figures. Do not simply copy something from a Web.}
\end{figure}



\section{Requirements}

%-------------------------------------REQUIREMENTS------------------------------------

\begin{enumerate}
	\item The game can be played on multiple Android devices (at least two).
	\item The gameplay should work as follows:
	\begin{enumerate}
		\item Coins are distributed evenly on the game map (board).
		\item One player plays as PacMan
		\item One or multiple other players play as ghosts
		\item PacMan wins, if he collects all coins on the map
		\item The ghosts win, if they capture PacMan (simply modeled by collision).
	\end{enumerate}
	\item Each player must use one Android device in order to control his figure (PacMan or ghost)
	\item The map (board) on which the players move should provide the following features:
	\begin{enumerate}
		\item PacMan starts on a predefined location (PacMan spawn)
		\item The ghosts (one ore multiple) start on predefined locations (ghost spawns)
		\item Player figures can only move up, down, left and right.
		\item The only structuring elements of the map are walls.
		\item Walls have eiter horizontal or vertical orientation.
		\item Player figures can not move through walls.
		\item The map has rectangular shape and is limited by walls at its borders (horizontal walls along the left and right border, vertical walls along upper and bottom border).
		\item Every position at the map that is not occupied by a wall must be reachable for the player figures. 
	\end{enumerate}
	\item The protocol used for Device-to-Device communication will will be implemented atop UDP, TCP or a higher-level protocol [???]
	\item The app will be optimized for Android version [???] and run on Android version [???] and higher. [???]
	\item The player who hosts the game will also play as PacMan.
	\item On the hosting player's device, a server task will be started that clients can connect to. 
	\item The players that connect to a hosted game play as ghosts.
	\item The game engine is either implemented from scratch using the android.graphics library [1] or the ANDEngine is used, a free Android 2D OpenGL game engine [2].
	
\end{enumerate}

\section{Work Packages}

%-----------------------------------WORK PACKAGES----------------------------------

\begin{itemize}
        \item {\bf WP1}: Define appropriate model for game situation (state) and methods to change the state.
        \item {\bf WP2}: Implement a map generator.         
        \item {\bf WP3}: Design graphical representation for PacMan player figure.
        \item {\bf WP4}: Design graphical representation for Ghost player figure.
        \item {\bf WP5}: Design graphical representation for map elements (coins, walls, unoccupied positions).
        \item {\bf WP6}: Implement Game activity that connects model and graphical representation. It should also allow the user to control it's figure.
		\item {\bf WP7}: Define a communication protocol to synchronize the game's state accross devices. Choose appropriate transport protocol. 
        \item {\bf WP8}: Implement the communication protocol - part 1: Server.
        \item {\bf WP9}: Implement the communication protocol - part 2: Client.
        \item {\bf WP10}: Design and implement start menu that allows the players to connect to the game.
        \item {\bf WP11}: Combine Game activity with the communication protocol.
\end{itemize}

\section{Milestones}

%-----------------------------------Milestones-------------------------------------

Work package distribution: 
\begin{itemize}
        \item {\bf Stefan Oanceas}:
		\item {\bf Johannes Beck}:
		\item {\bf Linus Fessler}:
		\item {\bf Markus Han}:
		\item {\bf Tiziano Zaramoni}:
		\item {\bf Florian Morath}:
\end{itemize}


%-------------------------------------REFERENCES-----------------------------------

% The following two commands are all you need in the
% initial runs of your .tex file to
% produce the bibliography for the citations in your paper.
\bibliographystyle{abbrv}
\bibliography{report}  % sigproc.bib is the name of the Bibliography in this case
% You must have a proper ".bib" file

%\balancecolumns % GM June 2007

\end{document}
